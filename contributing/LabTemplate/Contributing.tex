\lab{Template Lab}{Template Lab}
\objective{Outline the various guidelines and practices of writing a new lab.}
\label{lab:template}

\section{General Guidelines}
\begin{itemize}
\item No datasets or other data files should be committed to the repository.  This makes the repository unnecessarily large.  Data files are distributed via other means.  This is a strictly enforced requirement.  Even if you mistakenly commit a dataset and remove it later, git records the file in its history.  Any pull requests with data sets in the history will not be accepted.
\item Do not commit large images to the 
\item Do not use a real person's name in the lab manuals without their consent.  Common names are acceptable, but respect the privacy of others.
\end{itemize}


\section{\LaTeX File}
There are several conventions that we expect new labs to adhere to in regard to their usage of \LaTeX.

The written lab should use professional language with correct spelling and grammar.
Do not write labs using an overtly casual tone.

\subsection{General}
\begin{itemize}
\item Labs have a unique label (see labeling conventions)
\item Objective is coherent and interesting to read. Do not restate the title.
\end{itemize}

\subsection{\LaTeX}
\begin{itemize}
\item Use \texttt{\\[\\]} for \texttt{equation} environments for one line equations.  Never use \$\$ ... \$\$ for math mode.
\item Use \$ ... \$ for inline equations
\item \emph{Do not} use \texttt{\\frac} for inline equations.  One half should be represented at $1/2$. \emph{Do use} fractions for the other math environments.
\item Do not include editor/user specific lines in your \LaTeX document.
\item Use the global math commands that are defined for all the lab manuals.  The most common commands are: \texttt{\\norm}, \texttt{\\abs}, \texttt{\\set}, and \texttt{\\setconstruct}.
\item You are permitted to define variables and commands inside your lab \emph{only} if you undefine them at the end of you lab.  If you think your new command may be useful for more than just your lab, you may add it to the project wide \texttt{command.tex} file.
\item Labels follow a format of <type:name>.  The accepted types of labels are: 'fig' for figures, 'eqn' for equations, 'lab' for labs, 'table' for tables, 'prob' for problems, 'lst' for listings, 'sec' for sections.
\end{itemize}


\subsection{Code}
\begin{itemize}
\item Use listings environment only for code
\item Use pseudocode environments where needed.  Do not use listings environments or itemized lists for pseudocode.
\item Use \texttt{>>>} for all inputs inside listings environments.
\end{itemize}

\subsection{Figures and Tables}
\begin{itemize}
\item Don't force position.  The only instance we have encountered so far where position must be enforced is if your figure or table is contained in a problem environment.
\item Width is at most \texttt{\\textwidth}.
\item For multiple figures, use a subfigure environment.
\end{itemize}


\subsection{Plots}
Any code written to generate a plot for a lab must be included in the repository.
The code should be placed in a file named \texttt{plots.py}.
It should have the following format.
\lstinputlisting{plots.py}

This project uses a custom \texttt{matplotlibrc} file that defines the default attributes for all plots generated by Matplotlib.
All plots should use these defaults.
These defaults were designed to make plot look consistent across the project and to aid visibility when printed.
The resource file is compatible with Matplotlib 1.3.0 and later.
To use the project's \texttt{matplotlibrc} file, you load it as follows.
\begin{lstlisting}
import matplotlib
matplotlib.rcParams = matplotlib.rc_params_from_file('../../matplotlibrc')
\end{lstlisting}
This must be done before importing \li{matplotlib.pyplot}.

\subsection{Problems}
\begin{itemize}
\item Each problem should be unambiguously stated.
\item The answer of the problem must be easily checked.
\item The solution to the problem must be reproducible.  Do not rely random numbers unless you fix the seed.
\item Each problem should be uniquely labeled across the entire lab volume.
\end{itemize}

\section{Python Code}
There are a few guidelines that attempt to make the code in the project well-documented, consistent, and readable.

All code in the project must adhere to PEP8 (\url{http://legacy.python.org/dev/peps/pep-0008/}).

\begin{itemize}
\item No code should be executed on import
\item Importing scipy and pylab is highly discouraged.  Use the functions from their proper location.
\item All code must run cross-platform without modification
\item All libraries used must be available for Linus, OSX, and Windows.
\item All code must be properly organized and decomposed in appropriate files, classes, and functions.
\item Any global or lazy import are strictly forbidden
\item No time consuming code should run on import
\end{itemize}

\subsection{Solutions}
\begin{itemize}
\item Function names must correspond to the problem label.  Example: <prob:newton_1d> should have a corresponding solution with a function name of \li{def newton_1d(...)}.
\item Each function should be documented in an unambiguous manner.  All inputs and outputs of the function must be detailed.
\item The solutions must be verifably correct.
\end{itemize}

\subsection{Plots.py}
\begin{itemize}
\item All plots must be wrapped in functions
\item No interactive plotting features of matplotlib should be used.  Never use the \li{show()} method on a plot.  Always write to plot directly to file.
\item Always import the project's \texttt{matplotlibrc} file in the first two lines of code
\item Always output plots to a vector pdf with a filename that agrees with the function name and the plot label.
\end{itemize}



